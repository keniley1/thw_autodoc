%% Generated by Sphinx.
\def\sphinxdocclass{report}
\documentclass[letterpaper,10pt,english]{sphinxmanual}
\ifdefined\pdfpxdimen
   \let\sphinxpxdimen\pdfpxdimen\else\newdimen\sphinxpxdimen
\fi \sphinxpxdimen=.75bp\relax

\usepackage[utf8]{inputenc}
\ifdefined\DeclareUnicodeCharacter
 \ifdefined\DeclareUnicodeCharacterAsOptional
  \DeclareUnicodeCharacter{"00A0}{\nobreakspace}
  \DeclareUnicodeCharacter{"2500}{\sphinxunichar{2500}}
  \DeclareUnicodeCharacter{"2502}{\sphinxunichar{2502}}
  \DeclareUnicodeCharacter{"2514}{\sphinxunichar{2514}}
  \DeclareUnicodeCharacter{"251C}{\sphinxunichar{251C}}
  \DeclareUnicodeCharacter{"2572}{\textbackslash}
 \else
  \DeclareUnicodeCharacter{00A0}{\nobreakspace}
  \DeclareUnicodeCharacter{2500}{\sphinxunichar{2500}}
  \DeclareUnicodeCharacter{2502}{\sphinxunichar{2502}}
  \DeclareUnicodeCharacter{2514}{\sphinxunichar{2514}}
  \DeclareUnicodeCharacter{251C}{\sphinxunichar{251C}}
  \DeclareUnicodeCharacter{2572}{\textbackslash}
 \fi
\fi
\usepackage{cmap}
\usepackage[T1]{fontenc}
\usepackage{amsmath,amssymb,amstext}
\usepackage{babel}
\usepackage{times}
\usepackage[Bjarne]{fncychap}
\usepackage[dontkeepoldnames]{sphinx}

\usepackage{geometry}

% Include hyperref last.
\usepackage{hyperref}
% Fix anchor placement for figures with captions.
\usepackage{hypcap}% it must be loaded after hyperref.
% Set up styles of URL: it should be placed after hyperref.
\urlstyle{same}
\addto\captionsenglish{\renewcommand{\contentsname}{Contents:}}

\addto\captionsenglish{\renewcommand{\figurename}{Fig.}}
\addto\captionsenglish{\renewcommand{\tablename}{Table}}
\addto\captionsenglish{\renewcommand{\literalblockname}{Listing}}

\addto\captionsenglish{\renewcommand{\literalblockcontinuedname}{continued from previous page}}
\addto\captionsenglish{\renewcommand{\literalblockcontinuesname}{continues on next page}}

\addto\extrasenglish{\def\pageautorefname{page}}

\setcounter{tocdepth}{1}



\title{Sphinx Basics Documentation}
\date{Oct 17, 2018}
\release{0.1}
\author{The Hacker Within UIUC}
\newcommand{\sphinxlogo}{\vbox{}}
\renewcommand{\releasename}{Release}
\makeindex

\begin{document}

\maketitle
\sphinxtableofcontents
\phantomsection\label{\detokenize{index::doc}}



\chapter{test\_functions module}
\label{\detokenize{test_functions:test-functions-module}}\label{\detokenize{test_functions:module-test_functions}}\label{\detokenize{test_functions::doc}}\label{\detokenize{test_functions:welcome-to-sphinx-basics-s-documentation}}\index{test\_functions (module)}\index{mean\_val() (in module test\_functions)}

\begin{fulllineitems}
\phantomsection\label{\detokenize{test_functions:test_functions.mean_val}}\pysiglinewithargsret{\sphinxcode{test\_functions.}\sphinxbfcode{mean\_val}}{\emph{a}, \emph{b}}{}
This function returns the mean of arguments a and b: 0.5(a + b)
\begin{quote}\begin{description}
\item[{Parameters}] \leavevmode\begin{itemize}
\item {} 
\sphinxstyleliteralstrong{a} (\sphinxstyleliteralemphasis{float}) \textendash{} The first value.

\item {} 
\sphinxstyleliteralstrong{b} (\sphinxstyleliteralemphasis{float}) \textendash{} The second value.

\end{itemize}

\item[{Returns}] \leavevmode
The mean value.

\item[{Return type}] \leavevmode
float

\item[{Example}] \leavevmode
\fvset{hllines={, ,}}%
\begin{sphinxVerbatim}[commandchars=\\\{\}]
\PYG{g+gp}{\PYGZgt{}\PYGZgt{}\PYGZgt{} }\PYG{n}{mean\PYGZus{}val}\PYG{p}{(}\PYG{l+m+mi}{2}\PYG{p}{,} \PYG{l+m+mi}{5}\PYG{p}{)}
\PYG{g+go}{3.5}
\end{sphinxVerbatim}

\end{description}\end{quote}

\end{fulllineitems}

\index{square\_root() (in module test\_functions)}

\begin{fulllineitems}
\phantomsection\label{\detokenize{test_functions:test_functions.square_root}}\pysiglinewithargsret{\sphinxcode{test\_functions.}\sphinxbfcode{square\_root}}{\emph{val}, \emph{tol=0.0001}}{}
This is a docstring. Here I would explain what the square\_root() function does. (It calculates the square root. Duh.)
\begin{quote}\begin{description}
\item[{Parameters}] \leavevmode\begin{itemize}
\item {} 
\sphinxstyleliteralstrong{a} (\sphinxstyleliteralemphasis{float}) \textendash{} The value whose square root you want to calculate.

\item {} 
\sphinxstyleliteralstrong{tol} (\sphinxstyleliteralemphasis{float}) \textendash{} The tolerance of the solver. Smaller tolerance leads to higher precision. Default: 1e-4.

\end{itemize}

\item[{Returns}] \leavevmode
The square root of parameter \sphinxstyleemphasis{val}

\item[{Return type}] \leavevmode
float

\item[{Example}] \leavevmode
\fvset{hllines={, ,}}%
\begin{sphinxVerbatim}[commandchars=\\\{\}]
\PYG{g+gp}{\PYGZgt{}\PYGZgt{}\PYGZgt{} }\PYG{n}{value} \PYG{o}{=} \PYG{n}{square\PYGZus{}root}\PYG{p}{(}\PYG{l+m+mi}{4}\PYG{p}{)}
\PYG{g+gp}{\PYGZgt{}\PYGZgt{}\PYGZgt{} }\PYG{n+nb}{print}\PYG{p}{(}\PYG{n}{value}\PYG{p}{)}
\PYG{g+go}{2.0}
\end{sphinxVerbatim}

\end{description}\end{quote}

\end{fulllineitems}



\chapter{Indices and tables}
\label{\detokenize{index:indices-and-tables}}\begin{itemize}
\item {} 
\DUrole{xref,std,std-ref}{genindex}

\item {} 
\DUrole{xref,std,std-ref}{modindex}

\item {} 
\DUrole{xref,std,std-ref}{search}

\end{itemize}


\renewcommand{\indexname}{Python Module Index}
\begin{sphinxtheindex}
\def\bigletter#1{{\Large\sffamily#1}\nopagebreak\vspace{1mm}}
\bigletter{t}
\item {\sphinxstyleindexentry{test\_functions}}\sphinxstyleindexpageref{test_functions:\detokenize{module-test_functions}}
\end{sphinxtheindex}

\renewcommand{\indexname}{Index}
\printindex
\end{document}